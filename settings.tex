\documentclass[12pt]{article}
\usepackage[margin=1.5 in]{geometry}
% \usepackage[showframe]{geometry}
\usepackage{amsmath}
\usepackage{tcolorbox}
\usepackage{amssymb}
\usepackage{amsfonts}
\usepackage{amsthm}
\usepackage{lastpage}
\usepackage{hyperref}
\usepackage{fancyhdr}
\usepackage{accents}
\usepackage{titling}
\usepackage{polynom}
\usepackage{enumitem}
\usepackage{array}
\usepackage{setspace}
\usepackage{lipsum}
\renewcommand\maketitlehooka{\null\mbox{}\vfill}
\renewcommand\maketitlehookd{\vfill\null}
\pagestyle{fancy}
\setlength{\headheight}{40pt}
\newenvironment{solution}{
  \renewcommand\qedsymbol{$\blacksquare$}
  \begin{proof}[Solution]
  }{
  \end{proof}
  }
\newenvironment{sketch}{
  \renewcommand\qedsymbol{$\square$}
  \begin{proof}[Sketch]
  }{
  \end{proof}
}
\newenvironment{theorem}{
  \begin{proof}[Theorem]
  }{
  \end{proof}
}
\newenvironment{lemma}{
  \begin{proof}[Lemma]
  }{
  \end{proof}
}
\renewcommand\qedsymbol{$\blacksquare$}
\newcommand{\ubar}[1]{\underaccent{\bar}{#1}}
\newcommand{\Z}{\mathbb{Z}}
\newcommand{\R}{\mathbb{R}}
\newcommand{\Q}{\mathbb{Q}}
\newcommand{\C}{\mathbb{C}}
\newcommand{\N}{\mathbb{N}}
\newcommand{\charac}[1]{\mathrm{char}\,{#1}}
\newcommand{\idl}[1]{\langle {#1} \rangle}
\newcommand{\Mod}[1]{\, \mathrm{mod}\,{#1}}
\renewcommand{\deg}[1]{\mathrm{deg}{\,#1}}
\newcommand{\mult}[2]{\mathrm{mult} #1_{#2}}
\newcommand{\dsum}{\displaysyle \sum}
\newcommand{\ds}{\displaystyle}
\newcommand{\gen}[1]{\langle #1 \rangle}
\newcommand{\diam}[1]{\mathrm{diam}(#1)}
\newcommand{\rt}[1]{\,\sqrt[]{#1}}
\usepackage[labelformat=simple]{subcaption}
\renewcommand\thesubfigure{(\alph{subfigure})}
\newcommand{\Fex}[1]{F^{\tiny \mbox{ex}}_{#1}}
\usepackage{color}


\usepackage{url}
\usepackage{graphicx}
\usepackage[numbers,sort&compress]{natbib}

% \newtheorem{theorem}{Theorem}[section]
% \newtheorem{proposition}[theorem]{Proposition}
% \newtheorem{lemma}[theorem]{Lemma}
% \newtheorem{corollary}[theorem]{Corollary}
% \newtheorem{observation}[theorem]{Observation}

% \theoremstyle{definition}
% \newtheorem{definition}[theorem]{Definition}
% \newtheorem{algorithm}[theorem]{Algorithm}
% \newtheorem{conjecture}[theorem]{Conjecture}
% \newtheorem{question}[theorem]{Question}
% \newtheorem{problem}[theorem]{Problem}
% \newtheorem{exercise}[theorem]{Exercise}
% \newtheorem{result}[theorem]{Result}

\newtheorem{examplex}{Example}[section]
\newenvironment{example}{
\pushQED{\qed}
\renewcommand{\qedsymbol}{$\diamondsuit$}\examplex}{
  \popQED\endexamplex
}

\theoremstyle{remark}
\newtheorem{remark}{Remark}[section]


% Graph Theory

\newcommand{\ceeil}[1]{\left \lceil #1 \right \rceil }
\newcommand{\alg}{Shortlex Assignment Algorithm}
\newcommand{\red}[1]{\textcolor{red}{#1}}
\newcommand{\ecc}[2]{\mathrm{ecc}(#1, S_{#2})}
\newcommand{\compdiam}[1]{\mathrm{comp\,diam}(#1)}

% Tikz (for drawing diagrams.)

\usepackage{tikz}
\usetikzlibrary{calc}
\usetikzlibrary{arrows}
\usetikzlibrary{decorations.markings}
\tikzset{->-/.style={decoration={
			markings,
			mark=at position #1 with {\arrow{>}}},postaction={decorate}}}

\tikzstyle{vertex}=[circle, draw, inner sep=0pt, minimum size=4pt,fill=black]
\newcommand{\vertex}{\node[vertex]}
\tikzstyle{hollowvertex}=[circle, draw, inner sep=0pt, minimum size=4pt, fill=white]
\newcommand{\hollowvertex}{\node[hollowvertex]}
\tikzstyle{phantomvertex}=[circle, draw, inner sep=0pt, minimum size=4pt,color=white]
\newcommand{\phantomvertex}{\node[phantomvertex]}
	
\usepackage{enumitem}
\usepackage[vlined, ruled]{algorithm2e}
